%==============================================================================
% Elementos Pré-Textuais do TCC 
%------------------------------------------------------------------------------

% comando para mostrar a capa -------------------------------------------------
\imprimircapa

% inserindo folha de rosto ----------------------------------------------------
\imprimirfolhaderosto*

% ficha catalográfica ---------------------------------------------------------
% ############################ ATENÇÃO! #######################################
% Retire o comentário do código abaixo para inserir o PDF da ficha em questão.
% Depois APAGUE o código que se encontra da linha 22 até a 52
% 
%\begin{fichacatalografica}
%  \includepdf{fig_ficha_catalografica.pdf}
%\end{fichacatalografica}
%
%##############################################################################

\begin{fichacatalografica}
	\sffamily
	\vspace*{\fill}					% Posição vertical
	\begin{center}					% Minipage Centralizado
	\fbox{\begin{minipage}[c][8cm]{13.5cm}		% Largura
	\small
	\imprimirautor
	%Sobrenome, Nome do autor
	
	\hspace{0.5cm} \imprimirtitulo  / \imprimirautor. --
	\imprimirlocal, \imprimirdata-
	
	\hspace{0.5cm} \thelastpage p. : il. (algumas color.) ; 30 cm.\\
	
	\hspace{0.5cm} \imprimirorientadorRotulo~\imprimirorientador\\
	
	\hspace{0.5cm}
	\parbox[t]{\textwidth}{\imprimirtipotrabalho~--~\imprimirinstituicao,
	\imprimirdata.}\\
	
	\hspace{0.5cm}
		1. Palavra-chave1.
		2. Palavra-chave2.
		2. Palavra-chave3.
		I. Orientador.
		II. Universidade Federal do Recôncavo da Bahia.
		III. Centro de Formação de Professores.
		IV. Título 			
	\end{minipage}}
	\end{center}
\end{fichacatalografica}
%##############################################################################

% caso queira inserir alguma errata -------------------------------------------
%
% \begin{errata}
%   Colocar aqui o conteúdo da errata
% \end{errata}

% folha de aprovação ----------------------------------------------------------
% ############################ ATENÇÃO! #######################################
% Retire o comentário do código abaixo para inserir o PDF da folha de aprovação.
% Depois APAGUE o código que se encontra da linha 71 até a 105
% 
% \begin{folhadeaprovacao}
%   \includepdf{folhadeaprovacao_final.pdf}
% \end{folhadeaprovacao}
%
%##############################################################################
\begin{folhadeaprovacao}

  \begin{center}
    {\ABNTEXchapterfont\large\imprimirautor}

    \vspace*{\fill}\vspace*{\fill}
    \begin{center}
      \ABNTEXchapterfont\bfseries\Large\imprimirtitulo
    \end{center}
    \vspace*{\fill}
    
    \hspace{.45\textwidth}
    \begin{minipage}{.5\textwidth}
        \imprimirpreambulo
    \end{minipage}%
    \vspace*{\fill}
   \end{center}
        
   Trabalho aprovado. \imprimirlocal, xx de xxxxxxxxx de 202x:

   \assinatura{\textbf{\imprimirorientador} \\ Orientador} 
   \assinatura{\textbf{Professor} \\ Convidado 1}
   \assinatura{\textbf{Professor} \\ Convidado 2}
   %\assinatura{\textbf{Professor} \\ Convidado 3}
   %\assinatura{\textbf{Professor} \\ Convidado 4}
      
   \begin{center}
    \vspace*{0.5cm}
    {\large\imprimirlocal}
    \par
    {\large\imprimirdata}
    \vspace*{1cm}
  \end{center}
  
\end{folhadeaprovacao}
%##############################################################################

% dedicatória -----------------------------------------------------------------
\begin{dedicatoria}
  \vspace*{\fill}
  \centering
  \noindent
  \textit
  { 
    Este trabalho é dedicado a \\
    Thamyres, amada esposa; \\
    Asaph e Sophia, amados filhos.
  } 
  \vspace*{\fill}
\end{dedicatoria}

% agradecimentos %-------------------------------------------------------------

\begin{agradecimentos}
  \lipsum[1-3]    
\end{agradecimentos}

% epígrafe --------------------------------------------------------------------

\begin{epigrafe}
  \vspace*{\fill}
  \begin{flushright}
    \textit
    {
      ``Não vos amoldeis às estruturas deste mundo, \\
      mas transformai-vos pela renovação da mente, \\
      a fim de distinguir qual é a vontade de Deus: \\
      o que é bom, o que Lhe é agradável, o que é perfeito.''\\
      (Bíblia Sagrada, Romanos 12, 2)
    }
  \end{flushright}
\end{epigrafe}

% resumos ---------------------------------------------------------------------

% resumo em português
\setlength{\absparsep}{18pt} % ajusta o espaçamento dos parágrafos do resumo
\begin{resumo}
 Segundo a NBR6028:2003, o resumo deve ressaltar o objetivo, o método, os 
 resultados e as conclusões do documento. 
 A ordem e a extensão destes itens dependem do tipo de resumo (informativo ou 
 indicativo) e do tratamento que cada item recebe no documento original. 
 O resumo deve ser precedido da referência do documento, com exceção do resumo 
 inserido no próprio documento. 
 (\ldots) As palavras-chave devem figurar logo abaixo do resumo, antecedidas da 
 expressão Palavras-chave:, separadas entre si por ponto e finalizadas também 
 por ponto.

 \textbf{Palavras-chave}: latex. abntex2. modelo tcc. ufrb. cfp
\end{resumo}

% resumo em inglês
\begin{resumo}[Abstract]
 \begin{otherlanguage*}{english}
   This is the english abstract.

   \vspace{\onelineskip}
 
   \noindent 
   \textbf{Keywords}: latex. abntex2. modelo tcc. ufrb. cfp
 \end{otherlanguage*}
\end{resumo}


% inserindo lista de ilustrações ----------------------------------------------

\pdfbookmark[0]{\listfigurename}{lof}
\listoffigures*
\cleardoublepage

% inserindo lista de quadros --------------------------------------------------

\pdfbookmark[0]{\listofquadrosname}{loq}
\listofquadros*
\cleardoublepage

% inserindo lista de tabelas --------------------------------------------------

\pdfbookmark[0]{\listtablename}{lot}
\listoftables*
\cleardoublepage

% inserindo lista de abreviaturas e siglas ------------------------------------

\begin{siglas}
  \item[ABNT] Associação Brasileira de Normas Técnicas
  \item[abnTeX] ABsurdas Normas para TeX
  \item[CFP] Centro de Formação de Professores
  \item[UFRB] Universidade Federal do Recôncavo da Bahia
\end{siglas}

% inserin lista de símbolos ---------------------------------------------------

\begin{simbolos}
  \item[$ \mathbb{R} $] Conjunto dos Números Reais
  \item[$ \mathbb{C} $] Conjunto dos Números Complexos
  \item[$ \mathrm{Re}(z) $] Parte Real do complexo $ z $
  \item[$ \mathrm{sen} (x) $] Seno de $ x $ 
\end{simbolos}
% ---

% inserindo o sumário ---------------------------------------------------------
\pdfbookmark[0]{\contentsname}{toc}
\tableofcontents*
\cleardoublepage

%==============================================================================