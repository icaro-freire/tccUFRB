% arara: lualatex: {options: ['--output-directory=docs_saida']}
% arara: biber:    {options: ['--output-directory=docs_saida']}
% arara: lualatex: {options: ['--output-directory=docs_saida']}
% arara: lualatex: {options: ['--output-directory=docs_saida'], synctex: yes}

\documentclass
[
  12pt,      %--------------------------------------> tamanho da fonte do texto
  openright, %-------------------------> capítulo sempre começa em página ímpar
  twoside,   %--------------> configura a margem para imprimir "frente e verso"
  a4paper,   %-----------------------------------------------> tamanho do papel
  brazil     %------------------------------------> linguage principal do texto
]{cls/tccUFRB}

% pacotes, informações e configurações do TCC ---------------------------------
% pacotes utilizados =========================================================

% fontes & cia
\usepackage{euler} %------------------------------------> bela fonte matemática
\usepackage{fontspec} %-----------------------> codificação e escolha de fontes
\usepackage{libertine} %----------------> família de fontes (serif, sans, mono)

% matemática
\usepackage{amsmath, amssymb, amsthm} %--------------> escrita matemática (AMS)
\usepackage{mathtools} %---------------------------------> melhorias no amsmath

% cores
\usepackage{xcolor} %---->-------------------------------- modificação de cores

% figuras e tabelas
\usepackage{graphicx} %----------------------------------> inclusão de gráficos
  \graphicspath{{./figs/}} %-----------------------------> caminho para figuras

% metadados e testes
\usepackage{lipsum} %--------------------------> texto para testes tipográficos
\usepackage{hyperref} %---------------------------> para hiperlinks e metadados

% biblioigrafia/referências
\usepackage[style=abnt, justify]{biblatex} %----------------> para bibliografia
  \addbibresource{bib/biblatex-abnt.bib} %-> adiciona o arquivo da bibliografia
 %----------------> incluindo pacotes
% Informações de dados para CAPA e FOLHA DE ROSTO =============================
\titulo{Modelo Canônico de \\ Trabalho Acadêmico CFP/UFRB}
\autor{Seu Nome e Sobrenome}
\local{Amargosa~-~BA}
\data{\the\year}
\orientador{Google}
\coorientador{Modelo \abnTeX}
\instituicao{%
  Universidade Federal do Recôncavo da Bahia
  \par
  Centro de Formação de Professores
  \par
  Licenciatura em Matemética}
\tipotrabalho{Monografia (Graduação)}
% O preambulo deve conter o tipo do trabalho, o objetivo, 
% o nome da instituição e a área de concentração 
\preambulo{
  Monografia apresentada ao Curso de Licenciatura em Matemática do Centro de 
  Formação de Professores, da Universidade Federal do Recôncavo da Bahia, 
  Campus Amargosa, como requisito parcial para obtenção do título de Licenciado~(a) 
  em Matemática.
}

%============================================================================== %---------------> informações do TCC
%==============================================================================
% Configurações Gerais do TCC
%------------------------------------------------------------------------------

% configurando metadados do pdf -----------------------------------------------
% tonalidade do azul
\definecolor{blue}{RGB}{41,5,195}

% informações do PDF
\makeatletter
\hypersetup{
  %pagebackref=true,
	pdftitle       = {\@title}, 
	pdfauthor      = {\@author},
  pdfsubject     = {\imprimirpreambulo},
	pdfcreator     = {arara com LuaLaTeX},
	pdfkeywords    = {abnt}{latex}{abntex2}{biblatex}{ufrb}{cpf}{modelo tcc}, 
	colorlinks     = true,
  linkcolor      = blue,
  citecolor      = blue,
  filecolor      = red,
	urlcolor       = blue,
	bookmarksdepth = 4
}
\makeatother

% configurações de figuras e tabelas ------------------------------------------
% 																				"Posiciona figuras e tabelas no topo 
% 																				da página quando adicionadas sozinhas
% 																				em um página em branco." (abnTeX2)
\makeatletter
  \setlength{\@fptop}{5pt}
\makeatother

% criação de quadros e lista de quadros ---------------------------------------
% 																						"Possibilita criação de Quadros e 
% 																						Lista de quadros." (abnTeX2)
\newcommand{\quadroname}{Quadro}
\newcommand{\listofquadrosname}{Lista de quadros}

\newfloat[chapter]{quadro}{loq}{\quadroname}
\newlistof{listofquadros}{loq}{\listofquadrosname}
\newlistentry{quadro}{loq}{0}

% ajustes para ABNT
\setfloatadjustment{quadro}{\centering}
\counterwithout{quadro}{chapter}
\renewcommand{\cftquadroname}{\quadroname\space} 
\renewcommand*{\cftquadroaftersnum}{\hfill--\hfill}

\setfloatlocations{quadro}{hbtp}

% configurando espaçamentos ---------------------------------------------------
% sangria do parágrafo
\setlength{\parindent}{1.3cm}

%espaçamento entre linhas
\setlength{\parskip}{0.2cm}

% criação do índice -----------------------------------------------------------
\makeindex

%============================================================================== %-------------> configurações do TCC

%==============================================================================
% Início do Documento
%------------------------------------------------------------------------------

\begin{document}
%
  \selectlanguage{brazil} %--------------------------------> seleciona a língua
  \frenchspacing %------------------> elimina espaços em brancos inconvenientes 
%
% inserindo elementos pré-textuais --------------------------------------------
%  
  %==============================================================================
% Elementos Pré-Textuais do TCC 
%------------------------------------------------------------------------------

% comando para mostrar a capa -------------------------------------------------
\imprimircapa

% inserindo folha de rosto ----------------------------------------------------
\imprimirfolhaderosto*

% ficha catalográfica ---------------------------------------------------------
% ############################ ATENÇÃO! #######################################
% Retire o comentário do código abaixo para inserir o PDF da ficha em questão.
% Depois APAGUE o código que se encontra da linha 22 até a 52
% 
%\begin{fichacatalografica}
%  \includepdf{fig_ficha_catalografica.pdf}
%\end{fichacatalografica}
%
%##############################################################################

\begin{fichacatalografica}
	\sffamily
	\vspace*{\fill}					% Posição vertical
	\begin{center}					% Minipage Centralizado
	\fbox{\begin{minipage}[c][8cm]{13.5cm}		% Largura
	\small
	\imprimirautor
	%Sobrenome, Nome do autor
	
	\hspace{0.5cm} \imprimirtitulo  / \imprimirautor. --
	\imprimirlocal, \imprimirdata-
	
	\hspace{0.5cm} \thelastpage p. : il. (algumas color.) ; 30 cm.\\
	
	\hspace{0.5cm} \imprimirorientadorRotulo~\imprimirorientador\\
	
	\hspace{0.5cm}
	\parbox[t]{\textwidth}{\imprimirtipotrabalho~--~\imprimirinstituicao,
	\imprimirdata.}\\
	
	\hspace{0.5cm}
		1. Palavra-chave1.
		2. Palavra-chave2.
		2. Palavra-chave3.
		I. Orientador.
		II. Universidade Federal do Recôncavo da Bahia.
		III. Centro de Formação de Professores.
		IV. Título 			
	\end{minipage}}
	\end{center}
\end{fichacatalografica}
%##############################################################################

% caso queira inserir alguma errata -------------------------------------------
%
% \begin{errata}
%   Colocar aqui o conteúdo da errata
% \end{errata}

% folha de aprovação ----------------------------------------------------------
% ############################ ATENÇÃO! #######################################
% Retire o comentário do código abaixo para inserir o PDF da folha de aprovação.
% Depois APAGUE o código que se encontra da linha 71 até a 105
% 
% \begin{folhadeaprovacao}
%   \includepdf{folhadeaprovacao_final.pdf}
% \end{folhadeaprovacao}
%
%##############################################################################
\begin{folhadeaprovacao}

  \begin{center}
    {\ABNTEXchapterfont\large\imprimirautor}

    \vspace*{\fill}\vspace*{\fill}
    \begin{center}
      \ABNTEXchapterfont\bfseries\Large\imprimirtitulo
    \end{center}
    \vspace*{\fill}
    
    \hspace{.45\textwidth}
    \begin{minipage}{.5\textwidth}
        \imprimirpreambulo
    \end{minipage}%
    \vspace*{\fill}
   \end{center}
        
   Trabalho aprovado. \imprimirlocal, xx de xxxxxxxxx de 202x:

   \assinatura{\textbf{\imprimirorientador} \\ Orientador} 
   \assinatura{\textbf{Professor} \\ Convidado 1}
   \assinatura{\textbf{Professor} \\ Convidado 2}
   %\assinatura{\textbf{Professor} \\ Convidado 3}
   %\assinatura{\textbf{Professor} \\ Convidado 4}
      
   \begin{center}
    \vspace*{0.5cm}
    {\large\imprimirlocal}
    \par
    {\large\imprimirdata}
    \vspace*{1cm}
  \end{center}
  
\end{folhadeaprovacao}
%##############################################################################

% resumos ---------------------------------------------------------------------

% resumo em português
\setlength{\absparsep}{18pt} % ajusta o espaçamento dos parágrafos do resumo
\begin{resumo}
 Segundo a NBR6028:2003, o resumo deve ressaltar o objetivo, o método, os 
 resultados e as conclusões do documento. 
 A ordem e a extensão destes itens dependem do tipo de resumo (informativo ou 
 indicativo) e do tratamento que cada item recebe no documento original. 
 O resumo deve ser precedido da referência do documento, com exceção do resumo 
 inserido no próprio documento. 
 (\ldots) As palavras-chave devem figurar logo abaixo do resumo, antecedidas da 
 expressão Palavras-chave:, separadas entre si por ponto e finalizadas também 
 por ponto.

 \textbf{Palavras-chave}: latex. abntex2. modelo tcc. ufrb. cfp
\end{resumo}

% resumo em inglês
\begin{resumo}[Abstract]
 \begin{otherlanguage*}{english}
   This is the english abstract.

   \vspace{\onelineskip}
 
   \noindent 
   \textbf{Keywords}: latex. abntex2. modelo tcc. ufrb. cfp
 \end{otherlanguage*}
\end{resumo}


% inserindo o sumário ---------------------------------------------------------
\pdfbookmark[0]{\contentsname}{toc}
\tableofcontents*
\cleardoublepage

%============================================================================== %--> capa, folha de rosto, resumo ...
%
% inserindo elementos textuais ------------------------------------------------
%  
  \textual 
% 
  \chapter{Introdução}

\lipsum[1-8] %------------------------>  introdução
%
  \part{Fundamentação Teórica}
%
  \chapter{Um Tema Realmente Importante}

\lipsum[1-10] %---------------> capítulo 01
  \chapter{Algum Nome Interessante}

\lipsum[10-20] %---------------> capítulo 02
%
  \part{Resultados da Pesquisa}
%
  \chapter{Tenha criatividade}

\lipsum[1-10] %---------------> capítulo 03
% 
  \phantompart
% 
  \chapter{Conclusão}

\lipsum[31-33] %-------------------------> conclusão
%
% inserindo elementos pós-textuais --------------------------------------------
% 
  \postextual
%
% bibliografia
  \nocite{*} %------------------------> mostra referências usadas indiretamente
  \printbibliography %---------------------------------> imprime a bibliografia
%
% apêndices
  %==============================================================================
% Arquivo para incluir os APÊNDICES
%------------------------------------------------------------------------------

\begin{apendicesenv}
% 
  \partapendices % --------------------------------------> página dos apêndices
%
  %==============================================================================
% Apêndice 01
%------------------------------------------------------------------------------

\chapter{Primeiro Apêndice}
 
  \lipsum[1-10]

%=============================================================================


  %==============================================================================
% Apêndice 02
%------------------------------------------------------------------------------

\chapter{Mais Nomes Criativos}
 
  \lipsum[15-25]

%=============================================================================  
% 
\end{apendicesenv}

%==============================================================================  
%
% anexos
  %==============================================================================
% Arquivo para incluir os ANEXOS
%------------------------------------------------------------------------------

\begin{anexosenv}
% 
  \partanexos %---------------------------------> página para início dos anexos
%
  %==============================================================================
% Anexo 01
%------------------------------------------------------------------------------

\chapter{Primeiro Anexo}
 
  \lipsum[7-19]

%=============================================================================

%
\end{anexosenv}

%============================================================================== 
 
%------------------------------------------------------------------------------
\end{document}
%==============================================================================
